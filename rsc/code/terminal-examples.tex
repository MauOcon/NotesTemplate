\chapter{Terminal Examples}

This chapter demonstrates various terminal command examples using the custom terminal style.

\section{Basic Commands}

\begin{lstlisting}[style=terminal, caption=File Operations]
\$ ls -la
total 64
drwxr-xr-x  8 user user 4096 Jan 15 10:30 .
drwxr-xr-x  3 user user 4096 Jan 15 10:25 ..
-rw-r--r--  1 user user  220 Jan 15 10:30 .bashrc
-rw-r--r--  1 user user  807 Jan 15 10:30 .profile
drwxr-xr-x  2 user user 4096 Jan 15 10:30 Documents
drwxr-xr-x  2 user user 4096 Jan 15 10:30 Downloads

\$ mkdir new-project
\$ cd new-project
\$ touch README.md
\$ echo "Hello World" > hello.txt
\$ cat hello.txt
Hello World
\end{lstlisting}

\section{System Administration}

\begin{lstlisting}[style=terminal, caption=Package Management]
\$ sudo apt update
Hit:1 http://archive.ubuntu.com/ubuntu jammy InRelease
Get:2 http://archive.ubuntu.com/ubuntu jammy-updates InRelease
Reading package lists... Done
Building dependency tree... Done

\$ sudo apt install git curl wget
Reading package lists... Done
Building dependency tree... Done
The following NEW packages will be installed:
  curl git wget
0 upgraded, 3 newly installed, 0 to remove

\$ which git
/usr/bin/git
\$ git --version
git version 2.34.1
\end{lstlisting}

\section{Development Tools}

\begin{lstlisting}[style=terminal, caption=Git Operations]
\$ git init
Initialized empty Git repository in /home/user/project/.git/

\$ git add .
\$ git commit -m "Initial commit"
[main (root-commit) a1b2c3d] Initial commit
 2 files changed, 10 insertions(+)
 create mode 100644 README.md
 create mode 100644 hello.txt

\$ git remote add origin https://github.com/user/repo.git
\$ git push -u origin main
Enumerating objects: 4, done.
Counting objects: 100% (4/4), done.
Writing objects: 100% (4/4), 285 bytes | 285.00 KiB/s, done.
Total 4 (delta 0), reused 0 (delta 0), pack-reused 0
\end{lstlisting}

\section{Docker Commands}

\begin{lstlisting}[style=terminal, caption=Docker Operations]
\$ docker --version
Docker version 20.10.21, build baeda1f

\$ docker pull nginx:latest
latest: Pulling from library/nginx
a2abf6c4d29d: Pull complete
a9edb18cadd1: Pull complete
589b7251471a: Pull complete
Status: Downloaded newer image for nginx:latest

\$ docker run -d --name web-server -p 8080:80 nginx:latest
f1a2b3c4d5e6f7g8h9i0j1k2l3m4n5o6p7q8r9s0t1u2v3w4x5y6z7

\$ docker ps
CONTAINER ID   IMAGE          COMMAND                  STATUS
f1a2b3c4d5e6   nginx:latest   "/docker-entrypoint.…"   Up 2 minutes

\$ curl http://localhost:8080
<!DOCTYPE html>
<html>
<head>
<title>Welcome to nginx!</title>
</head>
<body>
<h1>Welcome to nginx!</h1>
</body>
</html>
\end{lstlisting}

\section{Network and Process Management}

\begin{lstlisting}[style=terminal, caption=System Monitoring]
\$ ps aux | grep nginx
root      1234  0.0  0.1  12345  6789 ?        Ss   10:30   0:00 nginx: master
www-data  1235  0.0  0.0  12345  6789 ?        S    10:30   0:00 nginx: worker

\$ netstat -tlnp | grep :80
tcp        0      0 0.0.0.0:80              0.0.0.0:*               LISTEN      1234/nginx

\$ top -n 1 | head -10
top - 10:35:42 up  2:15,  1 user,  load average: 0.08, 0.03, 0.01
Tasks: 234 total,   1 running, 233 sleeping,   0 stopped,   0 zombie
%Cpu(s):  1.2 us,  0.8 sy,  0.0 ni, 97.8 id,  0.2 wa,  0.0 hi,  0.0 si
MiB Mem :   7982.1 total,   5234.2 free,   1456.8 used,   1291.1 buff/cache
MiB Swap:   2048.0 total,   2048.0 free,      0.0 used.   6234.5 avail Mem

  PID USER      PR  NI    VIRT    RES    SHR S  %CPU  %MEM     TIME+ COMMAND
 1234 root      20   0   12345   6789   5678 S   0.0   0.1   0:00.12 nginx
 1235 www-data  20   0   12345   6789   5678 S   0.0   0.1   0:00.05 nginx
\end{lstlisting}

\section{File Search and Text Processing}

\begin{lstlisting}[style=terminal, caption=Search and Text Operations]
\$ find . -name "*.txt" -type f
./hello.txt
./docs/readme.txt
./logs/error.txt

\$ grep -r "Hello" .
./hello.txt:Hello World
./docs/readme.txt:Hello, this is a readme file

\$ sed 's/Hello/Hi/g' hello.txt
Hi World

\$ awk '{print NR ": " \$0}' hello.txt
1: Hello World

\$ wc -l *.txt
      1 hello.txt
      5 readme.txt
      6 total
\end{lstlisting}

\section{Archive and Compression}

\begin{lstlisting}[style=terminal, caption=Archive Operations]
\$ tar -czf backup.tar.gz Documents/
\$ ls -lh backup.tar.gz
-rw-r--r-- 1 user user 1.2M Jan 15 10:40 backup.tar.gz

\$ tar -tzf backup.tar.gz | head -5
Documents/
Documents/file1.txt
Documents/file2.txt
Documents/subfolder/
Documents/subfolder/file3.txt

\$ unzip -l archive.zip
Archive:  archive.zip
  Length      Date    Time    Name
---------  ---------- -----   ----
     1234  01-15-2024 10:30   file1.txt
     5678  01-15-2024 10:31   file2.txt
---------                     -------
     6912                     2 files
\end{lstlisting}

\section{Environment and Variables}

\begin{lstlisting}[style=terminal, caption=Environment Management]
\$ echo \$HOME
/home/user

\$ export MY_VAR="Hello Terminal"
\$ echo \$MY_VAR
Hello Terminal

\$ env | grep PATH
PATH=/usr/local/sbin:/usr/local/bin:/usr/sbin:/usr/bin:/sbin:/bin

\$ which python3
/usr/bin/python3

\$ python3 --version
Python 3.10.6

\$ pip3 list | head -5
Package    Version
---------- -------
certifi    2022.9.24
charset    4.0.0
idna       3.4
\end{lstlisting}

\section{Error Examples}

\begin{lstlisting}[style=terminal, caption=Common Errors and Solutions]
\$ cd /nonexistent
bash: cd: /nonexistent: No such file or directory

\$ cat missing-file.txt
cat: missing-file.txt: No such file or directory

\$ sudo systemctl status nonexistent-service
Unit nonexistent-service.service could not be found.

\$ docker run invalid-image
Unable to find image 'invalid-image:latest' locally
docker: Error response from daemon: pull access denied

\$ chmod 755 script.sh
\$ ./script.sh
Hello from script!
Script executed successfully.
\end{lstlisting}