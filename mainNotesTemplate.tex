\documentclass[10pt,letterpaper]{book}	
% babel: Este paquete establece las reglas tipográficas (y otras reglas) para el lenguaje especificado
\usepackage[spanish]{babel} 
%  inputenc: Indicamos la codificación que se utilizará en el documento por lo general ISO-8859-1(latin1)  o utf8
\usepackage[utf8]{inputenc} 
% Para escribir comillas dobles
\usepackage[style=mexican]{csquotes}
% Para indicar secuencias de menús o combinacines de teclas
\usepackage[os=mac, mackeys=symbols]{menukeys}
% Indicamos los margenes que tendrá el documento
\usepackage[top=1in, bottom=1.25in, left=0.75in, right=0.75in]{geometry}
% Permite el uso de unidades de medida
\usepackage{siunitx}
% amsmath: Comandos extras para matemáticas (cajas para ecuaciones, etc)
\usepackage{amsmath} 
% Simbolos matematicos (por lo tanto)
\usepackage{amssymb} 
% Incluir imágenes en LaTeX
\usepackage{graphicx}
% Para poder especificar el ancho de la tabla 
\usepackage{tabularx} 
%Podemos usar el especificador [H] en las figuras para que se queden donde queramos
\usepackage{float} 
 % Permite usar etiquetas fuera de elementos flotantes (etiquetas de figuras)
\usepackage{capt-of}
% Para que las referencias sean hipervínculos a las figuras o ecuaciones y aparezcan en color
\usepackage[colorlinks=true,plainpages=true,citecolor=blue,linkcolor=blue]{hyperref}
% Utilizar un glosario
\usepackage[automake]{glossaries-extra}
\newglossaryentry{Linux}
{
	name=linux,
	description={is a generic term referring to the family of Unix-like
		computer operating systems that use the Linux kernel},
	plural=linuces
}

\newacronym{page quality}{PQ}{Page Quality}
\makeglossaries
% Utilizar apéndices
\usepackage{appendix}
\renewcommand{\appendixname}{Apéndices}
\renewcommand{\appendixtocname}{Apéndices}
\renewcommand{\appendixpagename}{Apéndices} 

% Comando para poder cambiar el tamaño de las imágenes en un solo lugar
\newcommand{\GlLineWidth}{\textwidth}

%Paquete para listings-eclipse
\usepackage{rsc/pckg/lstcustom}

\author{Mauricio Ocón}
\title{Plantilla}

\begin{document}
	%Crea una hoja de portada
	\maketitle
	%Muestra el índice
	\tableofcontents	
	% Para que aparezcan las referencias sin citar
	\nocite{*}	
	
	%%%%%%% Capítulos %%%%%%%% 		
	\chapter{Preámbulo.}

\section{Configuración global del documento.}

\section{Paquetes}

\chapter{Contenido.}

\section{Imágenes}

	Lorem ipsum dolor sit amet, consectetur adipiscing elit. Nam viverra sagittis sapien, mollis ornare dolor. Donec fermentum, justo vitae tristique accumsan, urna ex dignissim risus, vitae tempor ante arcu condimentum turpis. Vestibulum tincidunt nulla nec lacus laoreet porttitor. Donec placerat congue mi, cursus dictum dolor viverra vitae. Phasellus venenatis, sapien in congue ornare, augue quam rutrum justo, vel molestie arcu metus luctus urna. Praesent egestas et leo vel ullamcorper. Etiam porta lectus dolor, vitae commodo arcu posuere in. Duis cursus suscipit lacus et tempus. Ut aliquet porttitor tortor in accumsan. Fusce pretium erat non ante tempor ultrices. Donec lorem justo, tincidunt ac finibus sit amet, tempor et dui. Fusce fermentum tempor nisl in blandit. Nam eu neque accumsan, varius tortor in, dictum justo. Ut rutrum tempus tellus, at sollicitudin velit ullamcorper non. In vel velit diam. Sed eu lectus iaculis, porta nisi sit amet, faucibus odio.
	
	Quisque rhoncus sit amet ex id porttitor. Aenean faucibus lectus at leo commodo feugiat. Mauris blandit ipsum non accumsan euismod. Cras nisi nunc, ultricies vel tellus non, ultrices iaculis mauris. Phasellus lorem lorem, cursus nec enim in, porta interdum arcu. Fusce ut imperdiet odio. Sed imperdiet eget enim at pellentesque.
	
	\begin{figure}[H]
		\begin{center}
			\includegraphics[width = 0.8\GlLineWidth]{rsc/img/TestImage}
			\captionof{figure}{\label{fig:testImgRef}Pie de imagen} 
		\end{center} 
	\end{figure}
	
	Quisque rhoncus sit amet ex id porttitor. Aenean faucibus lectus at leo commodo feugiat. Mauris blandit ipsum non accumsan euismod. Cras nisi nunc, ultricies vel tellus non, ultrices iaculis mauris. Phasellus lorem lorem, cursus nec enim in, porta interdum arcu. Fusce ut imperdiet odio. Sed imperdiet eget enim at pellentesque.
	
	Quisque rhoncus sit amet ex id porttitor. Aenean faucibus lectus at leo commodo feugiat. Mauris blandit ipsum non accumsan euismod. Cras nisi nunc, ultricies vel tellus non, ultrices iaculis mauris. Phasellus lorem lorem, cursus nec enim in, porta interdum arcu. Fusce ut imperdiet odio. Sed imperdiet eget enim at pellentesque.\cite{IEEEreferencias:Ref1} 
	
	Quisque rhoncus sit amet ex id porttitor. Aenean faucibus lectus at leo commodo feugiat. Mauris blandit ipsum non accumsan euismod. Cras nisi nunc, ultricies vel tellus non, ultrices iaculis mauris. Phasellus lorem lorem, cursus nec enim in, porta interdum arcu. Fusce ut imperdiet odio. Sed imperdiet eget enim at pellentesque.\gls{Linux}
	
	
	
\section{Tablas.}
	
	A continuación se muestra una tabla con las siguientes características
	\begin{itemize}
		\item Ancho igual al del texto.
		\item Se expecifica la alineación de las columnas
		\item Incluye un texto al pie de la tabla y una etiqueta para poder hacer referencia \ref{tab:tabularx_simple}.
	\end{itemize}
	
%	\begin{table}[H]
%		\begin{tabularx}{\textwidth}{|c|X|}
%			\hline 
%			\textbf{Núm. de pin} & \textbf{Descripción} \\ \hline 
%			1. GND & Se conecta a nuestra referencia de 0 \si{\volt} \\ \hline 
%			2. TX & Se conecta al pin RX de nuestra interfaz para configurar el módulo \\ \hline 
%			4. ENB & Se conecta a nuestro voltaje de alimentación de 3.3 \si{\volt} \\ \hline 
%			7. RX & Se conecta al pin RX de nuestra interfaz para configurar el módulo \\ \hline 
%			8. VCC & Se conecta a nuestro voltaje de alimentación a 3.3 \si{\volt}  \\ \hline 
%		\end{tabularx}
%		\caption{\label{tab:simple_tabularx} Pines que deberán conectarse al módulo para configurarlo.}
%	\end{table}
%	
%	También se pueden alinear las columnas como en la tabla \
%	
%	\begin{table}[H]	
%		\centering		
%		\begin{tabularx}{0.5\textwidth}{
%				|>{\raggedright\arraybackslash}X
%				|>{\centering\arraybackslash}X
%				|>{\raggedleft\arraybackslash}X|}
%			\hline 
%			$\phi$ & $\psi$ & $\phi\wedge\psi$ \\ \hline 
%			T & T & T \\ \hline 
%			T & F & F \\ \hline 
%			F & T & F \\ \hline 
%			F & F & F \\ \hline 
%		\end{tabularx} 
%		\caption{\label{tab:tabularx_align} Tabla de verdad proposicional. }
%	\end{table}
%	
%	\begin{tabularx}{0.7\textwidth}{
%			|>{\raggedright\arraybackslash}X
%			|>{\centering\arraybackslash}X
%			|>{\raggedleft\arraybackslash}X|
%	}
%		\hline
%		\multicolumn{1}{|c}{Título 1} & \multicolumn{1}{|c}{Título 2} & \multicolumn{1}{|c|}{Título 3} \\ \hline 
%		Contenido izquierda, contenido a la izquierda, contenido a la izquierda & Contenido izquierda & Contenido izquierda \\ \hline 
%		Más texto & Más texto & Más texto\\ \hline 
%	\end{tabularx}
	


	\begin{table}[H]
		\centering
	\begin{tblr}{
			width = 0.7\textwidth,
			colspec = {|X[l,m]|X[c,m]|X[r,m]|}, % Definimos la alineación en la especificación de columnas
			hlines,
			vlines,		
			rows = {m},
			row{odd} = {bg=gray!30},
			%row{even} = {bg=red!30},
			row{1} = {font=\bfseries, bg=gray!60, c}, % La primera fila centrada
		}	
		Encabezado 1 & Encabezado 2 & Encabezado 3 \\ \hline
		Texto izquierda & Texto centrado & Texto derecha \\
		Más contenido largo, más contenido largo, más contenido largo. & \colorbox{light-Green}{$\phi$} & 12345 \\
		Otro texto & \textcolor{Red}{texto} & Final	\\
		Texto izquierda & Texto centrado & Texto derecha \\
		Línea 1 \linebreak Línea 2 \linebreak Línea 3 & 
		Línea 1 \linebreak Línea 2 \linebreak Línea 3 & 
		Línea 1 \linebreak Línea 2 \linebreak Línea 3 \\		
		
	\end{tblr}
	\caption{\label{tab:simple_tabularx} Pines que deberán conectarse al módulo para configurarlo.}
	\end{table}
	
	Tabla con altura específica de filas \ref{tab:row_height}.
	
	\begin{table}[H]
		\begin{tblr}{
				%width = 0.7\textwidth,
				%colspec = {|X[1,l,m]|X[40,l,m]|}, % Definimos la alineación en la especificación de columnas
				%colspec = {|Q[l,m,10]|Q[c,m,90]},				
				colspec = {|c|X[l]|},			
				hlines,
				vlines,		
				rows = {m},
				row{odd} = {bg=gray!30},
				row{2-Z} = {4em},
				%row{even} = {bg=red!30},
				row{1} = {font=\bfseries, bg=gray!60, c}, % La primera fila centrada
			}	
			Exercise & English \\  
			(a) &   \\  
			(b) &   \\
			(c) &   \\
			(d) &   \\
			(e) &   \\
			(f) &   \\
		\end{tblr}
		\caption{\label{tab:row_height}}
	\end{table}

	
%	\begin{tblr}{
%			width = \textwidth,
%			colspec = {|X[c,m]|X[c,m]|X[c,m]|},
%			hlines,
%			vlines,
%			row{1} = {font=\bfseries, bg=lightgray},
%			rows = {m}, 
%			cell{2-last}{1} = {l},
%			cell{2-last}{3} = {r},
%			row{even} = {bg=lightgray!20}
%		}
%		Encabezado 1 & Encabezado 2 & Encabezado 3 \\
%		Texto izquierda & Texto centrado & Texto derecha \\
%		Más contenido & Con varias líneas & 12345 \\
%		Otro texto & Centrado & Final
%	\end{tblr}
	
	
	
	
	

	
	Vestibulum sollicitudin imperdiet nisl. Nullam eget nisi nunc. Quisque eget accumsan lorem, nec interdum neque. Pellentesque habitant morbi tristique senectus et netus et malesuada fames ac turpis egestas. Vestibulum ante ipsum primis in faucibus orci luctus et ultrices posuere cubilia Curae; Nullam at metus dignissim, euismod diam in, maximus sem. Sed semper rhoncus lectus eu tempor.
	
	Vestibulum sollicitudin imperdiet nisl. Nullam eget nisi nunc. Quisque eget accumsan lorem, nec interdum neque. Pellentesque habitant morbi tristique senectus et netus et malesuada fames ac turpis egestas. Vestibulum ante ipsum primis in faucibus orci luctus et ultrices posuere cubilia Curae; Nullam at metus dignissim, euismod diam in, maximus sem. Sed semper rhoncus lectus eu tempor.
	
	Vestibulum sollicitudin imperdiet nisl. Nullam eget nisi nunc. Quisque eget accumsan lorem, nec interdum neque. Pellentesque habitant morbi tristique senectus et netus et malesuada fames ac turpis egestas. Vestibulum ante ipsum primis in faucibus orci luctus et ultrices posuere cubilia Curae; Nullam at metus dignissim, euismod diam in, maximus sem. Sed semper rhoncus lectus eu tempor.
	
	\section{Sección 3}
	
	Phasellus bibendum dignissim efficitur. Vestibulum venenatis leo in lorem blandit dignissim. Praesent scelerisque dapibus lectus vel efficitur. Fusce tristique nulla at tempor eleifend. Vivamus ullamcorper volutpat lorem. Integer sodales molestie cursus. Aliquam felis mi, ultricies nec ante a, tincidunt semper nulla. In convallis finibus nibh id ultrices.

	\section{Sección 4}
	Quisque ligula ex, semper eu tristique sit amet, accumsan vitae eros. Proin sodales hendrerit metus sit amet tincidunt. Etiam dictum cursus tellus id malesuada. Integer accumsan augue eu lectus consectetur, vel interdum turpis posuere. Pellentesque pellentesque lacinia enim, semper consectetur sem lacinia in. Etiam bibendum tristique justo ac blandit. Curabitur vitae dapibus odio, tempus ornare dolor. Maecenas dui urna, luctus at auctor at, tincidunt non eros. Proin euismod, lacus nec faucibus congue, orci magna vulputate est, quis tempus orci urna eget justo. Nulla a placerat turpis.

	
	\chapter{Matemáticas.}

\large
\[
\textcolor{red}{\mathbf{Despite}} + 
\begin{aligned}[c]
	&\textcolor{green}{\mathbf{noun\;clause}}\\
	&\textcolor{green}{\mathbf{(not)\;-ing\;verb}}\\
	&\textcolor{violet}{\mathbf{the\;fact\;that}} + \textcolor{green}{\mathbf{clause}}
\end{aligned} + 
\mathbf{,} + \textcolor{cyan}{\mathbf{clause\;2}}
\]

\[
\textcolor{red}{\mathbf{Despite}} + 
\begin{aligned}[c]
	\textcolor{green}{\mathbf{noun\;clause}}\\
	\textcolor{green}{\mathbf{(not)\;-ing\;verb}}\\
	\textcolor{violet}{\mathbf{the\;fact\;that}} + \textcolor{green}{\mathbf{clause}}
\end{aligned} + 
\mathbf{,} + \textcolor{cyan}{\mathbf{clause\;2}}
\]

\[
\textcolor{red}{\mathbf{Despite}} + 
\begin{gathered}
	\textcolor{green}{\mathbf{noun\;clause}}\\
	\textcolor{green}{\mathbf{(not)\;-ing\;verb}}\\
	\textcolor{violet}{\mathbf{the\;fact\;that}} + \textcolor{green}{\mathbf{clause}}
\end{gathered} + 
\mathbf{,} + \textcolor{cyan}{\mathbf{clause\;2}}
\]



\section{Matrices}

\[
\begin{bNiceMatrix}[first-row,first-col]
	& A & B & C \\
	1 & \color{red}2 & 3 & 4 \\
	2 & 5 & \Block[fill=blue!20]{1-1}{6} & 7  \\
	3 & 8 & 9 & \Block[fill=green!30]{1-1}{10}
\end{bNiceMatrix}
\]

	

	\[
	\begin{bNiceMatrix}
		1 & 2 & 3 \\
		4 & 5 & 6 \\
		7 & 8 & 9
	\end{bNiceMatrix}
	\]
	
	% Matriz con corchetes personalizados
	\[
	\begin{pNiceMatrix}[margin]
		1 & 2 & 3 \\
		4 & 5 & 6 \\
		7 & 8 & 9
	\end{pNiceMatrix}
	\]
	
\[
\begin{bNiceMatrix}
	1 & \hspace*{1cm} & 0 \\
	& \Ddots^{n \text{ times}} & \\
	0 & & 1
\end{bNiceMatrix}
\]

\[
\begin{bNiceMatrix}
	a_1 & \Cdots & & & a_1 \\
	\Vdots & a_2 & \Cdots & & a_2 \\
	& \Vdots & \Ddots[color=red] \\
	\\
	a_1 & a_2 & & & a_n
\end{bNiceMatrix}
\]

\[
%\NiceMatrixOptions{renew-dots,renew-matrix}
\begin{pmatrix}
	1 & \cdots & \cdots & 1 \\
	0 & \ddots & & \vdots \\
	\vdots & \ddots & \ddots & \vdots \\
	0 & \cdots & 0 & 1
\end{pmatrix}
\]

\[
\begin{bNiceMatrix}
	1 & \hspace*{1cm} & 0 \\[8mm]
	& \Ddots[horizontal-label]^{n \text{ times}} & \\
	0 & & 1
\end{bNiceMatrix}
\]


\chapter{Programming tools}

Lorem ipsum dolor sit amet, consectetur adipiscing elit. Nam viverra sagittis sapien, mollis ornare dolor. Donec fermentum, justo vitae tristique accumsan, urna ex dignissim risus, vitae tempor ante arcu condimentum turpis. Vestibulum tincidunt nulla nec lacus laoreet porttitor. Donec placerat congue mi, cursus dictum dolor viverra vitae. Phasellus venenatis, sapien in congue ornare, augue quam rutrum justo, vel molestie arcu metus luctus urna. Praesent egestas et leo vel ullamcorper. Etiam porta lectus dolor, vitae commodo arcu posuere in. Duis cursus suscipit lacus et tempus. Ut aliquet porttitor tortor in accumsan. Fusce pretium erat non ante tempor ultrices. Donec lorem justo, tincidunt ac finibus sit amet, tempor et dui. Fusce fermentum tempor nisl in blandit. Nam eu neque accumsan, varius tortor in, dictum justo. Ut rutrum tempus tellus, at sollicitudin velit ullamcorper non. In vel velit diam. Sed eu lectus iaculis, porta nisi sit amet, faucibus odio.

\section{Listings}


	\subsection*{Listing Inline}
	
	% So we don't create weird spacing for single words.
	\lstset{breaklines=false}
	
	You can use the \ttfamily{\textbackslash listinline\$ listing text\$} command to use the custom
	eclipse style markup in paragraphs. To change the style of \ttfamily{\textbackslash listinline} command, you should alter the parameter \ttfamily{\textbackslash lstset} in \ttfamily{lstcustom.sty} file.  The 8 basic types in Java are
	\lstinline$boolean$, \lstinline$byte$, \lstinline$char$,
	\lstinline$short$, \lstinline$int$, \lstinline$long$,
	\lstinline$float$, and \lstinline$double$. These should have been
	properly formatted in the previous sentence.
	
	% Make sure to break lines the rest of the document
	\lstset{breaklines=true}
	
	\subsection*{Block Listing Example}
	
	A slightly longer block example is shown in listing \ref{lst:simple}.
	
	\begin{lstlisting}[caption={A simple listing.}, label={lst:simple}]
	/**
	* @param args
	*             Program arguments
	*/
	public static void main(String[] args) {
	// Now for the enlightening message.
	System.out.println("Hmm... hello big world!";
	// @TODO Finish this example
	// Just some comment that is probably too long to fit in the space provided....
	}
	\end{lstlisting}
	
	\subsection*{File Listing Example}
	
	An example that includes a \ttfamily{.java} file is shown in listing \ref{lst:file}.
	
	\lstinputlisting[caption={A listing from the file
		\ttfamily{Rectangle.java}}, label={lst:file}]{rsc/code/Rectangle.java}
	
	\lstinputlisting[caption={A listing from the file
		\ttfamily{Rectangle.java}}, label={lst:file}]{rsc/code/Person.java}
	
	\lstinputlisting[caption={A listing from the file
		\ttfamily{Rectangle.java}}, label={lst:file}, language=XML]{rsc/code/document.xml}
	
	\lstinputlisting[caption={A listing from the file
		\ttfamily{Rectangle.java}}, label={lst:file}, language=Groovy]{rsc/code/build.gradle}
	
\begin{lstlisting}[caption={Some Java code},label={lst:label},style=terminal]
select dd.data from dane dd 
\end{lstlisting}
		
\begin{lstlisting}[caption={A simple listing.}, label={lst:simple}, language=XML]
<project>
	<groupId>com.pluralsight</groupId>
	<artifactId>HelloWorld</artifactId>
	<version>1.0-SNAPSHOT</version>
	<modelVersion>4.0.0</modelVersion>
	<packaging>jar</packaging>
</project>
	\end{lstlisting}
\section{MenuKeys}

	\subsection{Menu}

	Para crear un nuevo proyecto, una vez abierto el Spring Tool Suite \menu{File > New > Spring Starter Project > ...}, como se muestra en la imagen. Si no aparece la opción Spring Starter Project, \menu{File > New > Other > Spring Boot > Spring Starter Project > ...}
	
	\subsection{Directory}
	
	\directory{C:/Users/mauoc/OneDrive/Notes/NotesTemplate}
	
	\subsection{keys}
	
	\keys{\ctrl+C} equals \keys{\ctrl + C}.
	
	

	
	\include{chp/chapter3}
	%%%%%%%%%%%%%%%%%%%%%%%%%%
	
	
	%%%%%%% Apéndices %%%%%%%%   	
	\appendix   
	\clearpage % o \cleardoublepage
	\addappheadtotoc 
	\appendixpage 
	%%%%%%%%%%%%%%%%%%%%%%%%%%
	
	%%%%%%% Anexos %%%%%%%% 
	\include{chp/anexos1}
	
	\include{chp/anexos2}
	%%%%%%%%%%%%%%%%%%%%%%%%%%
	
	\printglossaries
	
	%%%%%%% Bibliografía %%%%%%%%
	\bibliographystyle{bst/IEEEtran.bst} 
	\addcontentsline{toc}{chapter}{Referencias}  
	\bibliography{bib/IEEEabrv,bib/IEEEreferences.bib} 
	%%%%%%%%%%%%%%%%%%%%%%%%%%%%%
	
\end{document}