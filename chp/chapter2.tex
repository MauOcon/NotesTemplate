\chapter{Matemáticas.}

\large
\[
\textcolor{red}{\mathbf{Despite}} + 
\begin{aligned}[c]
	&\textcolor{green}{\mathbf{noun\;clause}}\\
	&\textcolor{green}{\mathbf{(not)\;-ing\;verb}}\\
	&\textcolor{violet}{\mathbf{the\;fact\;that}} + \textcolor{green}{\mathbf{clause}}
\end{aligned} + 
\mathbf{,} + \textcolor{cyan}{\mathbf{clause\;2}}
\]

\[
\textcolor{red}{\mathbf{Despite}} + 
\begin{aligned}[c]
	\textcolor{green}{\mathbf{noun\;clause}}\\
	\textcolor{green}{\mathbf{(not)\;-ing\;verb}}\\
	\textcolor{violet}{\mathbf{the\;fact\;that}} + \textcolor{green}{\mathbf{clause}}
\end{aligned} + 
\mathbf{,} + \textcolor{cyan}{\mathbf{clause\;2}}
\]

\[
\textcolor{red}{\mathbf{Despite}} + 
\begin{gathered}
	\textcolor{green}{\mathbf{noun\;clause}}\\
	\textcolor{green}{\mathbf{(not)\;-ing\;verb}}\\
	\textcolor{violet}{\mathbf{the\;fact\;that}} + \textcolor{green}{\mathbf{clause}}
\end{gathered} + 
\mathbf{,} + \textcolor{cyan}{\mathbf{clause\;2}}
\]



\section{Matrices}

\[
\begin{bNiceMatrix}[first-row,first-col]
	& A & B & C \\
	1 & \color{red}2 & 3 & 4 \\
	2 & 5 & \Block[fill=blue!20]{1-1}{6} & 7  \\
	3 & 8 & 9 & \Block[fill=green!30]{1-1}{10}
\end{bNiceMatrix}
\]

	

	\[
	\begin{bNiceMatrix}
		1 & 2 & 3 \\
		4 & 5 & 6 \\
		7 & 8 & 9
	\end{bNiceMatrix}
	\]
	
	% Matriz con corchetes personalizados
	\[
	\begin{pNiceMatrix}[margin]
		1 & 2 & 3 \\
		4 & 5 & 6 \\
		7 & 8 & 9
	\end{pNiceMatrix}
	\]
	
\[
\begin{bNiceMatrix}
	1 & \hspace*{1cm} & 0 \\
	& \Ddots^{n \text{ times}} & \\
	0 & & 1
\end{bNiceMatrix}
\]

\[
\begin{bNiceMatrix}
	a_1 & \Cdots & & & a_1 \\
	\Vdots & a_2 & \Cdots & & a_2 \\
	& \Vdots & \Ddots[color=red] \\
	\\
	a_1 & a_2 & & & a_n
\end{bNiceMatrix}
\]

\[
%\NiceMatrixOptions{renew-dots,renew-matrix}
\begin{pmatrix}
	1 & \cdots & \cdots & 1 \\
	0 & \ddots & & \vdots \\
	\vdots & \ddots & \ddots & \vdots \\
	0 & \cdots & 0 & 1
\end{pmatrix}
\]

\[
\begin{bNiceMatrix}
	1 & \hspace*{1cm} & 0 \\[8mm]
	& \Ddots[horizontal-label]^{n \text{ times}} & \\
	0 & & 1
\end{bNiceMatrix}
\]


\chapter{Programming tools}

Lorem ipsum dolor sit amet, consectetur adipiscing elit. Nam viverra sagittis sapien, mollis ornare dolor. Donec fermentum, justo vitae tristique accumsan, urna ex dignissim risus, vitae tempor ante arcu condimentum turpis. Vestibulum tincidunt nulla nec lacus laoreet porttitor. Donec placerat congue mi, cursus dictum dolor viverra vitae. Phasellus venenatis, sapien in congue ornare, augue quam rutrum justo, vel molestie arcu metus luctus urna. Praesent egestas et leo vel ullamcorper. Etiam porta lectus dolor, vitae commodo arcu posuere in. Duis cursus suscipit lacus et tempus. Ut aliquet porttitor tortor in accumsan. Fusce pretium erat non ante tempor ultrices. Donec lorem justo, tincidunt ac finibus sit amet, tempor et dui. Fusce fermentum tempor nisl in blandit. Nam eu neque accumsan, varius tortor in, dictum justo. Ut rutrum tempus tellus, at sollicitudin velit ullamcorper non. In vel velit diam. Sed eu lectus iaculis, porta nisi sit amet, faucibus odio.

\section{Listings}


	\subsection*{Listing Inline}
	
	% So we don't create weird spacing for single words.
	\lstset{breaklines=false}
	
	You can use the \ttfamily{\textbackslash listinline\$ listing text\$} command to use the custom
	eclipse style markup in paragraphs. To change the style of \ttfamily{\textbackslash listinline} command, you should alter the parameter \ttfamily{\textbackslash lstset} in \ttfamily{lstcustom.sty} file.  The 8 basic types in Java are
	\lstinline$boolean$, \lstinline$byte$, \lstinline$char$,
	\lstinline$short$, \lstinline$int$, \lstinline$long$,
	\lstinline$float$, and \lstinline$double$. These should have been
	properly formatted in the previous sentence.
	
	% Make sure to break lines the rest of the document
	\lstset{breaklines=true}
	
	\subsection*{Block Listing Example}
	
	A slightly longer block example is shown in listing \ref{lst:simple}.
	
	\begin{lstlisting}[caption={A simple listing.}, label={lst:simple}]
	/**
	* @param args
	*             Program arguments
	*/
	public static void main(String[] args) {
	// Now for the enlightening message.
	System.out.println("Hmm... hello big world!";
	// @TODO Finish this example
	// Just some comment that is probably too long to fit in the space provided....
	}
	\end{lstlisting}
	
	\subsection*{File Listing Example}
	
	An example that includes a \ttfamily{.java} file is shown in listing \ref{lst:file}.
	
	\lstinputlisting[caption={A listing from the file
		\ttfamily{Rectangle.java}}, label={lst:file}]{rsc/code/Rectangle.java}
	
	\lstinputlisting[caption={A listing from the file
		\ttfamily{Rectangle.java}}, label={lst:file}]{rsc/code/Person.java}
	
	\lstinputlisting[caption={A listing from the file
		\ttfamily{Rectangle.java}}, label={lst:file}, language=XML]{rsc/code/document.xml}
	
	\lstinputlisting[caption={A listing from the file
		\ttfamily{Rectangle.java}}, label={lst:file}, language=Groovy]{rsc/code/build.gradle}
	
\begin{lstlisting}[caption={Some Java code},label={lst:label},style=terminal]
select dd.data from dane dd 
\end{lstlisting}
		
\begin{lstlisting}[caption={A simple listing.}, label={lst:simple}, language=XML]
<project>
	<groupId>com.pluralsight</groupId>
	<artifactId>HelloWorld</artifactId>
	<version>1.0-SNAPSHOT</version>
	<modelVersion>4.0.0</modelVersion>
	<packaging>jar</packaging>
</project>
	\end{lstlisting}
\section{MenuKeys}

	\subsection{Menu}

	Para crear un nuevo proyecto, una vez abierto el Spring Tool Suite \menu{File > New > Spring Starter Project > ...}, como se muestra en la imagen. Si no aparece la opción Spring Starter Project, \menu{File > New > Other > Spring Boot > Spring Starter Project > ...}
	
	\subsection{Directory}
	
	\directory{C:/Users/mauoc/OneDrive/Notes/NotesTemplate}
	
	\subsection{keys}
	
	\keys{\ctrl+C} equals \keys{\ctrl + C}.
	
	
